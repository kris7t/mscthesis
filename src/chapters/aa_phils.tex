\chapter{Case study: dining philosophers}
\label{app:phils}

This case study presents the artifacts with their concrete textual syntax for
dining philosophers problem introduced in \vref{ex:background:running}.

We present the transformation \vref{ex:transform:feature} and \vref{ex:transform:mapping} with added symbols and annotations for reward analysis and mean-time-to-first-failure analysis with the PetriDotNet stochastic analysis tool.

\section{Graph queries}

Graph queries are written in the query language of \textabbr{VIATRA} Query~\citep{Ujhelyi15incquery}, a live model query evaluation tool. Due to the similarities with Datalog and other logical query languages, we follow the convention of starting variable names with a capital letter. Occasionally, this leads to a variable having the same name as its type; however, the result is not ambiguous. Class, reference and attribute names are highlighted in blue for clarity.

\lstinputlisting[style=block,language=vql,caption={Philosophers.vql}]{figures/Philosophers.vql}

\section{Petri net modules}

Petri net modules used the concrete syntax of our Ecore2Pn tool. Places, transitions and variables are highlighted in green, gold and blue, respectively. Names of references are written in italics.

The \textabbr{RGSPN} modules presented below extend the modules from \vref{ex:transform:mapping} with symbols for mean-time-to-first-failure analysis. The objective of this analysis is to determine the first philosopher to get hungry.

\lstinputlisting[style=block,language=ecore2pn,caption={PhilModule.ecore2pn}]{figures/PhilModule.ecore2pn}

\lstinputlisting[style=block,language=ecore2pn,caption={TableModule.ecore2pn}]{figures/TableModule.ecore2pn}

\section{Transformation specification}

We repeat the transformation specification from \vref{ex:transform:feature} and \vref{ex:transform:mapping} with the extensions for analyses performed with PetriDotNet.

The results of some analyses of models generated with this transformation were studied in the context of parameter identification of stochastic Petri nets by \citet[Chapter~5]{Molnar17optimization}.

\lstinputlisting[style=block,language=ecore2pn,caption={DiningPhilosophers.ecore2pn}]{figures/DiningPhilosophers.ecore2pn}